\documentclass[12pt,reqno]{amsart}
\usepackage[pdfborder={0 0 0.5 [3 2]}, plainpages=false]{hyperref}%
\usepackage[left=1in,right=1in,top=1in,bottom=1in]{geometry}%
\usepackage[citation-order]{amsrefs}%
\usepackage{amsmath}
\usepackage{enumerate}
\usepackage{amssymb}                
\usepackage{amsfonts}
\usepackage{amsthm}
\usepackage{bbm}
\usepackage[table,xcdraw]{xcolor}
\usepackage{float}
\usepackage{mathtools}
\usepackage{cool}
\usepackage{graphicx,epsfig}

\usepackage[capitalize,nameinlink]{cleveref}
% Per SIAM Style Manual, "section" should be lowercase
\crefname{section}{section}{sections}
\crefname{subsection}{subsection}{subsections}
\Crefname{section}{Section}{Sections}
\Crefname{subsection}{Subsection}{Subsections}

% Per SIAM Style Manual, "Figure" should be spelled out in references
\Crefname{figure}{Figure}{Figures}

% Per SIAM Style Manual, don't say equation in front on an equation.
\crefformat{equation}{\textup{#2(#1)#3}}
\crefrangeformat{equation}{\textup{#3(#1)#4--#5(#2)#6}}
\crefmultiformat{equation}{\textup{#2(#1)#3}}{ and \textup{#2(#1)#3}}
{, \textup{#2(#1)#3}}{, and \textup{#2(#1)#3}}
\crefrangemultiformat{equation}{\textup{#3(#1)#4--#5(#2)#6}}%
{ and \textup{#3(#1)#4--#5(#2)#6}}{, \textup{#3(#1)#4--#5(#2)#6}}{, and \textup{#3(#1)#4--#5(#2)#6}}

% But spell it out at the beginning of a sentence.
\Crefformat{equation}{#2Equation~\textup{(#1)}#3}
\Crefrangeformat{equation}{Equations~\textup{#3(#1)#4--#5(#2)#6}}
\Crefmultiformat{equation}{Equations~\textup{#2(#1)#3}}{ and \textup{#2(#1)#3}}
{, \textup{#2(#1)#3}}{, and \textup{#2(#1)#3}}
\Crefrangemultiformat{equation}{Equations~\textup{#3(#1)#4--#5(#2)#6}}%
{ and \textup{#3(#1)#4--#5(#2)#6}}{, \textup{#3(#1)#4--#5(#2)#6}}{, and \textup{#3(#1)#4--#5(#2)#6}}

% Make number non-italic in any environment.
\crefdefaultlabelformat{#2\textup{#1}#3}

\def\noi{\noindent}
\def\T{{\mathbb T}}
\def\R{{\mathbb R}}
\def\N{{\mathbb N}}
\def\C{{\mathbb C}}
\def\Z{{\mathbb Z}}
\def\P{{\mathbb P}}
\def\E{{\mathbb E}}
\def\Q{\mathbb{Q}}
\def\ind{{\mathbb I}}
\def\id{{\mathcal I}}
\def\per{\textrm{per}}

\DeclareMathOperator{\spn}{span}
\DeclareMathOperator{\ran}{range}

\graphicspath{ {images/} }

\newtheorem{lemma}{Lemma}
\newtheorem{theorem}{Theorem}
\newtheorem{corollary}{Corollary}
\newtheorem{definition}{Definition}
\newtheorem{proposition}{Proposition}
\newtheorem{hypothesis}{Hypothesis}
\newtheorem{remark}{Remark}

\newcommand{\revised}[1]{ \textcolor{red}{#1} }
\newcommand{\revisedd}[2]{ \textcolor{blue}{#1} }

\begin{document}

\title{Multi-breathers in the discrete sine-Gordon equation}

\author{Ross Parker}
\address{Department of Mathematics, Southern Methodist University, 
Dallas, TX 75275, USA}
\email{rhparker@smu.edu}

\author{P.\,G. Kevrekidis} 
\address{Department of Mathematics and Statistics, University of Massachusetts, Amherst MA 01003, USA}
\email{kevrekid@math.umass.edu}

\author{Alejandro Aceves}
\address{Department of Mathematics, Southern Methodist University, 
Dallas, TX 75275, USA}
\email{aaceves@smu.edu}

\begin{abstract}
	We consider the existence and spectral stability of multi-site breathers in the discrete Klein-Gordon equation.
\end{abstract}

\maketitle

\section{Introduction}

\section{Mathematical background}\label{sec:bg}

We will consider the discrete Klein-Gordon equation with on-site nonlinearity $f(u)$
\begin{equation}\label{eq:KG}
\ddot{u}_n = d (\Delta_2 u)_n - f(u_n),
\end{equation}
where $(\Delta_2 u)_n = u_{n+1} - 2 u_n + u_{n-1}$ is the discrete second difference operator, and $f(u) = P'(u)$ for a smooth potential function $P(u)$. 

\section{Exponential dichotomy}

CAPITAL LETTER CONVENTION?

\begin{equation}\label{eq:A0}
A_0 = \begin{pmatrix}
\dfrac{1}{d}\partial_t^2 + \dfrac{V''(0)}{d} + 2 & -\id \\ \id & 0
\end{pmatrix}
\end{equation}

\begin{lemma}\label{lemma:A0eigs}
The eigenvalues and corresponding eigenfunctions of $A_0$ are given by [INSERT HERE].
\end{lemma}
\begin{proof}
Consider the eigenvalue problem $A_0 u(t) = \lambda u(t)$ on [SPACE], where $u(t) = (v(t), w(t))^T$. We note that $\lambda = 0$ is not an eigenvalue, since that implies $v = w = 0$. The eigenvalue problem then reduces to the system of equations
\begin{align}\label{eq:A0EVPsystem}
\left( \frac{1}{d}\partial_t^2 + \frac{V''(0)}{d} + 2 \right) v(t) = \left( \lambda + \frac{1}{\lambda} \right) v(t), \quad
w = \frac{1}{\lambda} v(t).
\end{align}
Letting $r = \lambda + \frac{1}{\lambda}$ and using the periodic boundary conditions, the set of solutions to \cref{eq:A0EVPsystem} is given by
\begin{align}
v_k(t) &= \frac{1}{T} \exp\left( i \frac{2 \pi k t}{T} \right), \quad r_k = -\frac{4 k^2 \pi^2}{d T^2} + \frac{V''(0)}{d} + 2 && k \in \Z,
\end{align}
where the functions $v_k(t)$ have been normalized. The eigenvalues of $A_0$ are then given by $\left\{ \lambda_k, \lambda_k^{-1} \right\}$, where
\begin{equation}\label{eq:A0lambdak}
\lambda_k = \frac{1}{2}\left( r_k + \sqrt{r_k^2 - 4} \right),
\end{equation}
and the corresponding eigenfunctions are
\begin{align}
u_k(t) &= \begin{pmatrix}v_k(t) \\ \lambda_k^{-1}  v_k(t) \end{pmatrix}, \quad
u_k^{-1}(t) = \begin{pmatrix}v_k(t) \\ \lambda_k w_k(t) \end{pmatrix} & k \in \Z.
\end{align}
\end{proof}

It follows from \cref{lemma:A0eigs} that the spectrum of $A_0$ is bounded away from the unit circle provided $|r_k| > 2$ for all $k$. [MAYBE SOMETHING HERE ABOUT HOW THIS IS ALWAYS POSSIBLE FOR SUFFICIENTLY SMALL $d$]

\begin{lemma}\label{eq:A0basis}
[HYPERBOLICITY ASSUMPTION]
The set of eigenfunctions $\{ u_k(t), u_k(t) : k \in \Z \}$ of $A_0$ are a Riesz basis for [HILBERT SPACE], i.e. every function $y(t)$ can be written uniquely as
\begin{equation}\label{eq:yinA0basis}
y(t) = \sum_{k \in \Z} a_k u_k(t) + \sum_{k \in \Z} b_k u^{-1}_k(t),
\end{equation}
where $a_k, b_k \in \C$ and the sum converges.
\begin{proof}
Letting $v_k(t) = \frac{1}{T} \exp\left( i \frac{2 \pi k t}{T} \right)$, we note that the set $\{ z^1_k(t), z^2_k(t) : k \in \Z \}$, where $z^1_k(t) = (v_k(t), 0)^T$ and $z^2_k(t) = (0, v_k(t))^T$ is an orthonormal basis for [HILBERT SPACE]. Therefore, there exist unique scalars $c_k, d_k \in \C$ such that
\begin{equation*}
y(t) = \sum_{k \in \Z} c_k z^1_k(t) + \sum_{k \in \Z} d_k z^2_k(t),
\end{equation*}
and the sum converges. Then \cref{eq:yinA0basis} follows by taking
\[
a_k = \frac{1}{\lambda_k^2 - 1}\left(-c_k + \lambda_k d_k \right), \quad
b_k = \frac{1}{\lambda_k^2 - 1}\left( \lambda_k^2 c_k - \lambda_k d_k \right),
\]
where $\lambda_k$ is defined in \cref{eq:A0lambdak}, and $\lambda_k^2 \neq 1$ by [HYPERBOLICITY ASSUMPTION].
\end{proof}
\end{lemma}

\begin{theorem}The equation $u(n+1) = A_0 u(n)$ has exponential dichotomies on $\Z^+$ and $Z^-$.
\end{theorem}




\paragraph{Acknowledgments}

This material is based upon work supported by the U.S. National Science Foundation under the RTG grant DMS-1840260 (R.P. and A.A.)
and DMS-1809074 (P.G.K.).

\bibliographystyle{amsplain}
\bibliography{DKG.bib}

\end{document}